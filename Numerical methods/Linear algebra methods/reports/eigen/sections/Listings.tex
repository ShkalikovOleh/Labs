\section*{\centering Додатки}

\definecolor{deepblue}{rgb}{0,0,0.5}
\definecolor{deepred}{rgb}{0.6,0,0}
\definecolor{deepgreen}{rgb}{0,0.5,0}

\lstset{
    breaklines=true,
    postbreak=\mbox{\textcolor{red}{$\hookrightarrow$}\space},
    basicstyle=\ttfamily,
    keywordstyle=\color{blue}\ttfamily,
    stringstyle=\color{red}\ttfamily,
    commentstyle=\color{green}\ttfamily,
    morecomment=[l][\color{magenta}]{\#}
    showstringspaces=false
}

Далі наведено програмний код імплементованих алгоритмів.
Вихідний код, який було створено для даного практикума
(у тому числі \LaTeX),
можна знайти за наступним
\href{https://github.com/ShkalikovOleh/Programming-Labs}{посиланням}.

\subsubsection*{Класс для інкупсуляції роботи з пам'ятю}
\lstinputlisting[language=C++]{../../src/src/Buffer.h}

\subsubsection*{Клас векторів}
\lstinputlisting[language=C++]{../../src/src/Vector.h}

\subsubsection*{Клас матриць}
\lstinputlisting[language=C++]{../../src/src/Matrix.h}

\subsubsection*{Утілити для матриць}
\lstinputlisting[language=C++]{../../src/src/MatrixView.h}

\subsubsection*{Клас для роботи з поліномами}
\lstinputlisting[language=C++]{../../src/src/Polynomial.h}

\subsubsection*{Методи для обрахунку коренів поліномів}
\lstinputlisting[language=C++]{../../src/src/PolynomialSolvers.h}

\subsubsection*{Методи знаходження власних значень}
\lstinputlisting[language=C++]{../../src/src/Eigen.h}
