\section*{\centering Практична частина}

Розглянемо застосування ітераційного
метода Якобі на прикладі наступної матриці:

\[
    A = \begin{pmatrix}
        7 & 0.88 & 0.93 & 1.23 \\
        0.88 & 4.16 & 1.3 & 0.15 \\
        0.93 & 1.3 & 6.44 & 2 \\
        1.21 & 0.15 & 2 & 9
    \end{pmatrix}
\]
Наведемо таблицю зі значенням матиці повороту та сферичної норми($S$) матриці $A$
на кожній ітерації алгоритму. Зазначимо, що $T^{-1}=T^T$.

\begin{longtable}[h]{|c|c|c|c|c|}
        \hline
        \textbf{k} & Матриця $T_k$ & $S_d$ & $S_{nd}$ & $S$ \\
        \hline
        1
        & $
        \begin{pmatrix}
                1 & 0 & 0 & 0 \\
                0 & 1 & 0 & 0 \\
                0 & 0 & 0.877227 & 0.480076 \\
                0 & 0 & -0.480076 & 0.877227
        \end{pmatrix} $
        & 196.779
        & 9.6318
        & 206.411
        \\ \hline 2
        & $
        \begin{pmatrix}
                0.926325 & 0 & 0 & 0.376726 \\
                0 & 1 & 0 & 0 \\
                0 & 0 & 1 & 0 \\
                -0.376726 & 0 & 0 & 0.926325
        \end{pmatrix} $
        & 201.327
        & 5.08418
        & 206.411
        \\ \hline 3
        & $
        \begin{pmatrix}
                1 & 0 & 0 & 0 \\
                0 & 0.861724 & 0.507378 & 0 \\
                0 & -0.507378 & 0.861724 & 0 \\
                0 & 0 & 0 & 1
        \end{pmatrix} $
        & 203.61
        & 2.8013
        & 206.411
        \\ \hline 4
        & $
        \begin{pmatrix}
                1 & 0 & 0 & 0 \\
                0 & 0.993337 & 0 & 0.115243 \\
                0 & 0 & 1 & 0 \\
                0 & -0.115243 & 0 & 0.993337
        \end{pmatrix} $
        & 205.034
        & 1.37667
        & 206.411
        \\ \hline 5
        & $
        \begin{pmatrix}
                1 & 0 & 0 & 0 \\
                0 & 1 & 0 & 0 \\
                0 & 0 & 0.992703 & 0.120586 \\
                0 & 0 & -0.120586 & 0.992703
        \end{pmatrix} $
        & 205.744
        & 0.667086
        & 206.411
        \\ \hline 6
        & $
        \begin{pmatrix}
                0.512215 & 0 & 0.858857 & 0 \\
                0 & 1 & 0 & 0 \\
                -0.858857 & 0 & 0.512215 & 0 \\
                0 & 0 & 0 & 1
        \end{pmatrix} $
        & 206.146
        & 0.264725
        & 206.411
        \\ \hline 7
        & $
        \begin{pmatrix}
                1 & 0 & 0 & 0 \\
                0 & 0.996784 & 0.0801361 & 0 \\
                0 & -0.0801361 & 0.996784 & 0 \\
                0 & 0 & 0 & 1
        \end{pmatrix} $
        & 206.282
        & 0.128844
        & 206.411
        \\ \hline 8
        & $
        \begin{pmatrix}
                0.103824 & 0.994596 & 0 & 0 \\
                -0.994596 & 0.103824 & 0 & 0 \\
                0 & 0 & 1 & 0 \\
                0 & 0 & 0 & 1
        \end{pmatrix} $
        & 206.392
        & 0.0187992
        & 206.411
        \\ \hline 9
        & $
        \begin{pmatrix}
                1 & 0 & 0 & 0 \\
                0 & 1 & 0 & 0 \\
                0 & 0 & 0.999817 & 0.0191274 \\
                0 & 0 & -0.0191274 & 0.999817
        \end{pmatrix} $
        & 206.405
        & 0.0058601
        & 206.411
        \\ \hline 10
        & $
        \begin{pmatrix}
                1 & 0 & 0 & 0 \\
                0 & 0.999959 & 0 & 0.00901594 \\
                0 & 0 & 1 & 0 \\
                0 & -0.00901594 & 0 & 0.999959
        \end{pmatrix} $
        & 206.41
        & 0.00142838
        & 206.411
        \\ \hline 11
        & $
        \begin{pmatrix}
                0.999997 & 0 & 0 & 0.00264493 \\
                0 & 1 & 0 & 0 \\
                0 & 0 & 1 & 0 \\
                -0.00264493 & 0 & 0 & 0.999997
        \end{pmatrix} $
        & 206.41
        & 0.00064294
        & 206.411
        \\ \hline 12
        & $
        \begin{pmatrix}
                1 & 0 & 0 & 0 \\
                0 & 0.999845 & 0.0175874 & 0 \\
                0 & -0.0175874 & 0.999845 & 0 \\
                0 & 0 & 0 & 1
        \end{pmatrix} $
        & 206.411
        & 5.10303e-06
        & 206.411
        \\ \hline 13
        & $
        \begin{pmatrix}
                1 & 0 & 0.000479337 & 0 \\
                0 & 1 & 0 & 0 \\
                -0.000479337 & 0 & 1 & 0 \\
                0 & 0 & 0 & 1
        \end{pmatrix} $
        & 206.411
        & 1.39792e-07
        & 206.411
        \\ \hline 14
        & $
        \begin{pmatrix}
                1 & 9.08961e-05 & 0 & 0 \\
                -9.08961e-05 & 1 & 0 & 0 \\
                0 & 0 & 1 & 0 \\
                0 & 0 & 0 & 1
        \end{pmatrix} $
        & 206.411
        & 5.45773e-08
        & 206.411
        \\ \hline 15
        & $
        \begin{pmatrix}
                1 & 0 & 0 & 0 \\
                0 & 1 & 0 & 0 \\
                0 & 0 & 1 & 3.92664e-05 \\
                0 & 0 & -3.92664e-05 & 1
        \end{pmatrix} $
        & 206.411
        & 2.28514e-11
        & 206.411 \\
        \hline
\end{longtable}

Таким чином отримали наступні значення власних чисел:
\begin{align*}
    \lambda_1 = 3.38753 \hspace*{1cm} \lambda_2 = 5.65842 \\
    \lambda_3 = 6.67398 \hspace*{1cm} \lambda_4 = 10.8801
\end{align*}
