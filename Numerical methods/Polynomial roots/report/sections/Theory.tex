\section*{\centering Теоретичні відомості}

Основна задача даного лабораторного практикуму полягає у тому,
щоб дослідити методи розв'язання нелінійних рівнянь:
\[
    f(x) = 0
\]
, де $f(x)$ - неперервна, та для деяких методів дифференційовна функція.
Розв'язання даної задачі вимагає виконання 2 етапів:
\begin{enumerate}
    \item Відокремлення коренів - визначення проміжків $[a_i, b_i]$, у
    яких знаходиться лише 1 корінь.
    \item Уточнення кореня у кожному з проміжків, знайденних на попередному етапі.
\end{enumerate}
Для поліномів, тобто функції вигляду:
\begin{equation} \label{th:polynomial}
    f(x) = \sum\limits_{i=0}^n a_n x^n = a_0 + a_1x + \dots + a_nx^n
\end{equation}
задача значно спрощується, бо мають місце теореми, які дозволяють
оцінити значення коренів рівняння та відділити їх. Розглянемо деякі з них.

\begin{theorem}[про границі усіх коренів рівняння] \label{th:bounds}
    Нехай $A = \max |a_i|, i = \overline{0, n-1}$ та
    $A = \max |a_i|, i = \overline{1, n}$.
    Тоді всі корені рівняння \ref{th:polynomial} лежать у кільці:
    \[
        \frac{|a_0|}{B + |a_0|} \leq |x| \leq \frac{|a_n|+A}{|a_n|}
    \]
\end{theorem}

\begin{theorem}[Гюа] \label{th:gua}
    Якщо
    $ \exists k \in \overline{1, n-1} \; : a^2_k < a_{k-1} a_{k+1} $
    ,то рівняння \ref{th:polynomial} має хоча б одну пару комплексноспряжених коренів.
\end{theorem}

\begin{theorem}[про верхню межу додатніх коренів] \label{th:upper_bound}
    Нехай $B = \max |a_i|, a_i < 0$ та
    $m = \max i, a_i < 0 $.
    Тоді верхня межа додатніх коренів рівняння \ref{th:polynomial}:
    \[
        R = 1 + \sqrt[n-m]{\frac{B}{a_n}}
    \]
\end{theorem}

\begin{corollary}
    Очевидно що, якщо обрахувати значення $R_1$, $R_2$, $R_3$ та $R_4$,
    вважаючи за аргумент вирази $x$, $\dfrac{1}{y}$, $-\dfrac{1}{y}$, $-y$, то
    будуть мати місце наступні оцінки:
    \[
        \frac{1}{R_2} \leq x^{+} \leq R_1 \hspace{1cm}
        -R_4 \leq x^{-} \leq -\frac{1}{R_3}
    \]
\end{corollary}

\begin{corollary}[Спосіб Лагранжа] \label{th:lagrange}
    Нехай $f(x) = F(x) + H(x)$, причому $F(x)$
    містить усі поспіль старші члени $f$ з $a_i > 0$,
    а також усі члени, для яких $a_i < 0$, а $H(x)$
    усі інші члени. Тоді, якщо існує таке $\alpha>0$,
    що $F(\alpha) > 0$, то для коренів $x_r$ рівняння
    $f$ виконується: $x_r < \alpha$.
\end{corollary}

\begin{theorem}[Штурма] \label{th:sturm}
    Нехай $f(x) = P_n(x)$ поліном без кратних коренів. Утворимо послідовність
    многочленів:
    \[
        p_0(x) = f(x) \hspace{1cm}
        p_1(x) = f^{'}(x) \hspace{1cm}
        p_{i+1} = - (p_{i-1} \mod{p_i})
    \]
    Тоді кількість дійсних коренів полінома $f(x)$ на довільному відрізку $[a, b]$
    дорівнює різниці між кількістю змін знаку у цій послідовності при $x=a$ та $x=b$.
\end{theorem}

Для того, щоб приблизно обчислити значення кореня у
визначеному проміжку будемо використовувати наступні методи.

\begin{algorithm}[H] \label{alg:bisection}
    \SetAlgoLined
    \KwIn{$f$, $a$, $b$, $\varepsilon$}
    \While{$\varepsilon < |b-a|$}
    {
        $ c = \dfrac{a+b}{2} $\;

        \eIf{f(c)f(a) > 0}
        {
            $a \leftarrow c$\;
        }
        {
            $b \leftarrow c$\;
        }
    }
    \KwOut{$c$}
 \caption{Метод бісекцій}
\end{algorithm}

\begin{algorithm}[H] \label{alg:newton}
    \SetAlgoLined
    \KwIn{$f$, $x_0$, $\varepsilon$}
    $x_k \leftarrow x_0 $\;
    \While{$\varepsilon < f(x_k)$}
    {
        $x_{k+1} = x_k - \dfrac{f^{'}(x_k)}{f(x_k)}$ \;
    }
    \KwOut{$x_k$}
 \caption{Метод Ньютона}
\end{algorithm}


\begin{algorithm}[H] \label{alg:secant}
    \SetAlgoLined
    \KwIn{$f$, $a$, $b$, $\varepsilon$}
    \While{$\varepsilon < f(c)$}
    {
        $ c = \dfrac{af(b) - bf(a)}{f(b)-f(a)} $\;

        \eIf{f(c)f(a) > 0}
        {
            $a \leftarrow c$\;
        }
        {
            $b \leftarrow c$\;
        }
    }
    \KwOut{$c$}
 \caption{Метод хорд}
\end{algorithm}

% Для успішної роботи методу рекомендовано, щоб на відрізку,
% який містить шуканий корінь,
% функція була монотонна та друга її похідна не змінювала знак.
