\subsection*{Варіант 3}


\begin{center}
    \begin{tikzpicture}
    [node distance=1.3cm,on grid,>=stealth',bend angle=25,auto,
    every place/.style= {minimum size=6mm,thick,draw=blue!75,fill=blue!20},
    every transition/.style={thick,draw=black!75,fill=black!20},
    red place/.style= {place,draw=red!75,fill=red!20}]

    \node [place,tokens=1] at (-3, 0) (A2) {};
    \node [place] at (-3, 3) (B2) {};
    \node [place,tokens=1] at (-4.5, 1) (C2) {};

    \node [place,tokens=1] at (0, 0) (A3) {};
    \node [place] at (0, 3) (B3) {};
    \node [place,tokens=1] at (-1.5, 1) (C3) {};
    \node [place] at (0, 6) (E3) {};
    \node [place] at (1.5, 9) (D3) {};
    \node [place] at (-1.5, 9) (F3) {};

    \node [place,tokens=1] at (3, 0) (A4) {};
    \node [place] at (3, 3) (B4) {};
    \node [place,tokens=1] at (1.5, 1) (C4) {};

    \node [transition] at (0, 1.5) (f3) {}
    edge [pre] (A3)
    edge [pre, bend left] (C3)
    edge [post] (B3);

    \node [transition] at (0, 4.5) (r3) {}
    edge [pre] (B3)
    edge [pre] (B4)
    edge [post] (A4)
    edge [post] (E3);

    \node [transition] at (1.5, 10.5) (e3) {}
    edge [pre] (D3)
    edge [post, bend right=9] (C3)
    edge [post, bend left] (C4)
    edge [post, bend left] (A3);

    \node [transition] at (1.5, 7.5) (eat3) {}
    edge [pre] (E3)
    edge [post] (D3);

    \node [transition] at (-1.5, 7.5) (inv3) {}
    edge [pre, bend left=9] (A2)
    edge [pre] (B3)
    edge [post] (F3)
    edge [post, bend right=30] (A2);

    \node [transition] at (-1.5, 10.5) (einv3) {}
    edge [pre] (F3)
    edge [post, bend right] (C3)
    edge [post, bend right=25] (A3);


    \node [transition] at (3, 1.5) (f4) {}
    edge [pre] (A4)
    edge [pre, bend left] (C4)
    edge [post] (B4);

    \node [transition] at (-3, 1.5) (f2) {}
    edge [pre] (A2)
    edge [pre, bend left] (C2)
    edge [post] (B2);

    \end{tikzpicture}
\end{center}

Опишемо значення переходів та позицій цього варіанту мережі Петрі.
Позиції:
\begin{enumerate}
    \item Філософ гуляє
    \item Виделка лежить на своєму місці
    \item Філософ взяв ліву виделку
    \item Філософ взяв обидві виделки
    \item \begin{enumerate}
        \item (права гілка) Філософ обідає
        \item (ліва гілка) Філософ приєднався до тих, кто обідає(сидить за столом)
    \end{enumerate}
\end{enumerate}

Переходи:
\begin{enumerate}
    \item Філософ бере ліву виделки
    \item Філософ бере праву виделку(її передав йому правий сусід)
    \item \begin{enumerate}
        \item (права гілка) Філософ готується до обіду
        \item (ліва гілка) Філософ запрошується до столу
    \end{enumerate}
    \item Філософ повертається до прогулянки та кладе виделки(у) на місце
\end{enumerate}

Простими словами механізм запрошення можна описати так: запрошується той, хто
ні отримав праву виделку від сусіда, ні віддав свою виделку лівому сусіду.