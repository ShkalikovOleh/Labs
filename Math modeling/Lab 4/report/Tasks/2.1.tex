\subsection*{Варіант 1}

\begin{center}
    \begin{tikzpicture}
    [node distance=1.3cm,on grid,>=stealth',bend angle=25,auto,
    every place/.style= {minimum size=6mm,thick,draw=blue!75,fill=blue!20},
    every transition/.style={thick,draw=black!75,fill=black!20},
    red place/.style= {place,draw=red!75,fill=red!20}]

    \node [place,tokens=1] at (-6, 0) (A1) {};
    \node [place] at (-6, 3) (B1) {};
    \node [place,tokens=1] at (-7.5, 1) (C1) {};

    \node [place,tokens=1] at (-3, 0) (A2) {};
    \node [place] at (-3, 3) (B2) {};
    \node [place,tokens=1] at (-4.5, 1) (C2) {};

    \node [place,tokens=1] at (0, 0) (A3) {};
    \node [place] at (0, 3) (B3) {};
    \node [place,tokens=1] at (-1.5, 1) (C3) {};

    \node [place,tokens=1] at (3, 0) (A4) {};
    \node [place] at (3, 3) (B4) {};
    \node [place,tokens=1] at (1.5, 1) (C4) {};

    \node [place,tokens=1] at (6, 0) (A5) {};
    \node [place] at (6, 3) (B5) {};
    \node [place,tokens=1] at (4.5, 1) (C5) {};

    \node [transition] at (-6, 1.5) (f1) {}
    edge [pre] (A1)
    edge [pre, bend left] (C1)
    edge [pre, bend right] (C2)
    edge [post] (B1);

    \node [transition] at (-6, 4.5) (e1) {}
    edge [pre] (B1)
    edge [post] (C1)
    edge [post] (C2)
    edge [post, bend right] (A1);

    \node [transition] at (-3, 1.5) (f2) {}
    edge [pre] (A2)
    edge [pre, bend left] (C2)
    edge [pre, bend right] (C3)
    edge [post] (B2);

    \node [transition] at (-3, 4.5) (e2) {}
    edge [pre] (B2)
    edge [post] (C2)
    edge [post] (C3)
    edge [post, bend right] (A2);

    \node [transition] at (0, 1.5) (f3) {}
    edge [pre] (A3)
    edge [pre, bend left] (C3)
    edge [pre, bend right] (C4)
    edge [post] (B3);

    \node [transition] at (0, 4.5) (e3) {}
    edge [pre] (B3)
    edge [post] (C3)
    edge [post] (C4)
    edge [post, bend right] (A3);

    \node [transition] at (3, 1.5) (f4) {}
    edge [pre] (A4)
    edge [pre, bend left] (C4)
    edge [pre, bend right] (C5)
    edge [post] (B4);

    \node [transition] at (3, 4.5) (e4) {}
    edge [pre] (B4)
    edge [post] (C4)
    edge [post] (C5)
    edge [post, bend right] (A4);

    \node [transition] at (6, 1.5) (f5) {}
    edge [pre] (A5)
    edge [pre, bend left] (C5)
    edge [pre, bend left=50] (C1)
    edge [post] (B5);

    \node [transition] at (6, 4.5) (e5) {}
    edge [pre] (B5)
    edge [post] (C5)
    edge [post, bend right=25] (C1)
    edge [post, bend right] (A5);

    \end{tikzpicture}
\end{center}

Тут і надалі будемо нумерувати рівні позицій(кулі) та переходів(квадрати) знизу догори
(окремо позиції та переходи).
Тобто, вершини рівню 1 - це найнижчі 5 позицій у графі вище.
Позиції рівню 1 - це найнижчі 5 переходів у графі вище.

Опишемо значення переходів та позицій цього варіатну мережі Петрі.
Позиції:
\begin{enumerate}
    \item Філософ гуляє
    \item Виделка лежить на своєму місці
    \item Філософ обідає
\end{enumerate}

Переходи:
\begin{enumerate}
    \item Філософ бере 2 виделки
    \item Філософ повертається до прогулянки та кладе виделки на місце
\end{enumerate}
